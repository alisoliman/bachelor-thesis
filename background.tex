\chapter{Background}
\label{chap:background}

\section{Notification} 
\label{sec:s2}
While we think that turning off notifications or switching off our smartphones can lead for a better results, Experiments showed that in some of the cases it actually shows positive results while in other cases it showed more interruptions as some of the users were always refreshing waiting for the next. \cite{citeKey3}

Experiments also showed that notifications who calls for action and diverts to another application are most probably having a negative effect and a huge cost of interruption. A solution was proposed to that is to eliminate the call for action buttons and to make the notification only focus on the new information part and let the users decide whether they would like to navigate to another application or not. However, This didn't show an insightful indicator whether users were affected with this or not.

%The goal of this paper was to understand how interruptions caused by notifications affected the user's focus on his primary task. Making it more specific only email notifications were taken into consideration. 20 Users were involved in this experiment and they were monitored for 2 weeks.
%
%First week was a normal working week, Second week notifications were disabled. Two different behaviours were observed depending on the users after disabling the notifications. First, There were some users who had an increase in the rate of accessing their email client waiting for new emails. While the others experienced more focus and they visited their email client less than they used to. Post Study showed that despite the cost of these notifications users highly value the information that these notifications hold in hand and they are willing to invest their time in order to check the new information.
%
%This highlights the importance of our research and strongly supports it. We need to balance between the cost of notifications and the interruption it costs to achieve higher productivity rates. 

\section{Task Modelling}

Empirical studies were conducted and the authors decided to take psychological theories approach. To backup this approach a Task division model was implemented to tackle the problem.
\subparagraph{Task Division Model}
focused on categorizing our day to day tasks. Based on the users answer the author proposed two categories for the new Model.
\begin{itemize}
	\item Hard Subtasks - This includes but not limited to tasks that needs a lot of focus like brainstorming an idea or thinking about the architecture of a new built city.
	\item Easy Subtasks - This includes but not limited to tasks that the user knows how to execute exactly. Let's say that a new method is being implemented and I have done the very same task a couple of times previously and I know exactly how I'm going to do it this time. This can be modeled as an easy subtask.
\end{itemize}
\subparagraph{Primary}
tasks were divided into subtasks. Moreover, These subtasks were modeled based on categorizing these subtasks into the two above categories. Users were given large tasks and were asked to divide these tasks into subtasks and categorize them into the mentioned above two categories and what the majority agreed upon as a hard task was chosen as a spot that is the best fit for a notification interruption. This actually showed a positive impact on the users and they were able to focus more on their tasks and finish them in less time.\cite{citeKey5}

\section{Cost of Interruption}
As notifications have different cost of interruption this cost of interruption is inversely proportional to the urgency of the notification to the user, Users have different notification priorities. This is why a solution was proposed to overcome this issue by utilizing the same notification in different ways in terms of design. Delivering this notification will depend on the its urgency to the user. The author thought about attention as a constrained resource that can be traded for some utility. This utility is enabled by perceiving additional, valued information while performing other primary tasks. \cite{citeKey6}

\section{Bio Sensors}
Also Bio-Sensors have a huge advantage over other emotional detection methods as they can overcome many environmental conditions that challenges other emotional detection methods. Research results indicates that post bio-sensors training can achieve 89.9\% - 96.6\% accuracy rates of emotional state detection. \cite{citeKey4}